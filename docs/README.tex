\documentclass{article}
\usepackage{graphicx} % Required for inserting images
\usepackage{array}
\usepackage{booktabs}
\usepackage{listings}
\usepackage{booktabs}
\usepackage{array}
%\usepackage{fontspec} % for XeLaTeX/LuaLaTeX
% or
\usepackage{lmodern} % for pdfLaTeX
\usepackage{xcolor}
\definecolor{iris}{rgb}{0.35, 0.31, 0.81}
\definecolor{ivory}{rgb}{1.0, 1.0, 0.94}
\definecolor{jonquil}{rgb}{0.98, 0.85, 0.37}
\definecolor{Jade}{rgb}{0.0, 0.66, 0.42}
\definecolor{lavender}{rgb}{0.84, 0.79, 0.87}

\title{Biological Data Summary}
\author{Peter Shaw}
\date{August 2024}

\begin{document}
\lstset{
    language=Python,
    basicstyle=\small\ttfamily,
    numbers=left,
    numbersep=5pt,
    frame=single,
    captionpos=b,
    breaklines=true,
    breakatwhitespace=false
}

\maketitle

\section{Eye data}
\subsection{background}

\textbf{
Q? From a genetics perspective, please briefly describe what a DMR is?\\
}
\textbf{
Q? ok expand this definition to explain why you want 3D genetic data to find genes enhanced by a DMR?\\
}
{\color{iris} LLM generated

A Differentially Methylated Region (DMR) is a genomic region where DNA methylation levels differ significantly between two or more groups, such as different cell types, tissues, or disease states. DNA methylation typically acts to repress gene expression by preventing transcription factors from binding to the promoter regions of genes.

When we identify DMRs, it's often because we want to understand which genes are regulated by these epigenetic changes. However, DMRs can be located far away from the genes they regulate, and traditional 2D genomics approaches (e.g., ChIP-seq, bisulfite sequencing) may not provide enough information to connect a DMR to its target gene(s).

This is where 3D genetic data comes in!

Chromatin conformation capture techniques like Hi-C, Micro-C, or 4C allow us to study the three-dimensional organization of chromatin and identify long-range interactions between different genomic regions. These interactions can bring together distant regulatory elements (e.g., enhancers) with their target genes.

By integrating DMRs with 3D genetic data, we can:

Link DMRs to potential target genes : By identifying which genes are in close proximity to a DMR in the 3D space of chromatin, we can infer which genes might be regulated by that DMR.
Prioritize candidate regulatory elements : Hi-C and other 3D genomics data help identify topologically associating domains (TADs), which are self-interacting regions of chromatin that often harbor enhancer-promoter interactions. By finding DMRs within these TADs, we can focus on the most likely regulatory elements.
Enhance interpretation of GWAS results : Genome-wide association studies (GWAS) often identify variants associated with disease risk, but it's challenging to determine which genes are actually responsible for the observed associations. Integrating GWAS signals with DMRs and 3D genetic data can provide a more comprehensive understanding of how these variants affect gene regulation.
By leveraging 3D genetic data, we can gain insights into the complex relationships between epigenetic modifications (like DNA methylation), chromatin organization, and gene expression. This integrated approach helps us better understand the mechanisms underlying cellular behavior and disease pathogenesis.
}
\subsection{Tables}

We have two types of DRM data. One is computed using glm analysis (DSS1) and one using HOM1 neural networks. But both are essentially the same. The DSS spread sheet is likely more reliable. The main difference between the two tables is column Q. In the DSS data this is an AREA statistic, which is proportional to the length of each DRM. In the Home data, this is a confidence score. Both relate to the confidence of edges for that DRM.

Either of these tables can be used to produce a bipartite graph between DMRs and genes.
The DRM ID is in column D/E is the DMR ID. Each DMR connects to its closest gene (Column M in HOME1 and DSS1) and additional genes (which are obtained using 3D gene mapping and so further away > 2.5K) are listed in column R in DSS1 and column S in Home1. e.g. Rgs20/e5;Oprk1/e4. Importantly, note that e4 means enhancer e4. For example, when creating the bipartite network graph, DMR3 would only connect to Rgs20 $\cap$ Rgs20 and Oprk1. i.e. ignore the part after the / e4. When there is no enhancer data available, a period "." is placed in column S, and just the details in column M can be used.

We have no exact data for the enhancers in eyes yet. So the data in column 4 are taken from embryonic stem cell data.

In addition, column O contains some functional description. Potentially we could create a column for each functional description and match DMR to this column too.

\begin{table*}[h]
\centering
\sffamily
\setlength{\tabcolsep}{5pt}
\begin{tabular}
{@{}p{0.6cm}p{3.0cm}p{7.5cm}@{}}
\toprule
& \bf{Column Label} & \bf{Description} \\
\midrule
D & DMR ID & Unique identifier for each differentially methylated region (DMR) \\
M & Closest Gene & The gene closest to each DMR \\
Q & Area & Measure of the reliability of associations between DMRs and genes in DSS1 dataset \\
R & Additional Genes & Additional genes associated with each DMR, obtained through 3D gene mapping \\
S & ENCODE Promoter Interaction (BingRen Lab) & Enhancer information from ENCODE promoter interaction data \\
\bottomrule
\end{tabular}
\caption{Description of columns in the DSS1 dataset}
\label{table:dss1_columns}
\end{table*}

\section{Output Files}
The output files generated from the processing include:

\begin{itemize}
    \item \textbf{bipartite_graph_output.txt}: This file contains the edges of the bipartite graph, with the first line indicating the number of unique DMRs and genes. Each subsequent line represents a connection between a DMR and a gene, with the gene represented by its unique ID.
    \item \textbf{gene_ids.csv}: This CSV file contains a mapping of gene names to their unique IDs, allowing for easy reference and analysis of the genes associated with the DMRs.
    \item \textbf{bipartite_graph_output.txt.biclusters}: Contains the identified bicliques in the network, with each line representing the nodes (both DMRs and genes) that form a complete bipartite subgraph.
\end{itemize}

\subsection{Preliminary Results}
Initial analysis of the DSS1 dataset revealed detailed patterns in the DMR-gene interaction network:

\begin{itemize}
    \item \textbf{Network Coverage}:
    \begin{itemize}
        \item Complete coverage achieved for both DMRs (2,109/2,109) and genes (4,536/4,536)
        \item Edge analysis revealed:
        \begin{itemize}
            \item 20 edges with single coverage
            \item No edges with multiple coverage
            \item 12,083 uncovered edges
        \end{itemize}
    \end{itemize}
    
    \item \textbf{Biclique Distribution}:
    \begin{itemize}
        \item Total bicliques identified: 1,114
        \item Classification breakdown:
        \begin{itemize}
            \item 328 trivial bicliques (1 DMR, 1 gene)
            \item 628 small bicliques
            \item 158 interesting bicliques (≥3 DMRs, ≥3 genes)
        \end{itemize}
    \end{itemize}
    
    \item \textbf{Size Distribution Analysis}:
    \begin{itemize}
        \item Range spans from minimal 1×1 up to substantial 24×6 configurations
        \item Notable larger configurations include 13×17 and 10×28 bicliques
        \item Majority concentrated in smaller size ranges
    \end{itemize}
\end{itemize}

\begin{table}[h]
\centering
\begin{tabular}{lr}
\toprule
\textbf{Metric} & \textbf{Value} \\
\midrule
Total Bicliques & 1,114 \\
Trivial Bicliques (1×1) & 328 \\
Small Bicliques & 628 \\
Interesting Bicliques (≥3×3) & 158 \\
DMR Coverage & 100\% (2,109/2,109) \\
Gene Coverage & 100\% (4,536/4,536) \\
Single Coverage Edges & 20 \\
Uncovered Edges & 12,083 \\
\bottomrule
\end{tabular}
\caption{Detailed bicliques analysis results}
\label{table:biclique_results}
\end{table}

For each interesting biclique identified, detailed information was captured including:
\begin{itemize}
    \item DMR identifiers with corresponding area statistics
    \item Associated gene names and their functional descriptions
    \item Organized groupings of related DMRs and genes
\end{itemize}

These results reveal a complex regulatory network structure with distinct patterns of DMR-gene interactions. The presence of numerous interesting bicliques, particularly the larger configurations (e.g., 24×6, 13×17, 10×28), suggests coordinated regulatory relationships where groups of DMRs may collectively influence sets of functionally related genes. The high number of uncovered edges (12,083) might indicate either noise in the original network or weaker interactions that don't form complete bicliques.

\textbf{Note}: The HOME1 dataset analysis is pending and will provide additional validation of these patterns when completed.

\section{Bicliques Analysis}
The analysis identifies bipartite cliques (bicliques) in the DMR-gene interaction network. A biclique represents a complete subgraph where every DMR is connected to every gene within that subgraph.

\subsection{Classification}
Bicliques are classified into three categories:
\begin{itemize}
    \item \textbf{Trivial}: Contains exactly 1 DMR and 1 gene
    \item \textbf{Small}: Contains either 2 DMRs or 2 genes (or both)
    \item \textbf{Interesting}: Contains 3 or more DMRs and 3 or more genes
\end{itemize}

\subsection{Analysis Output}
The bicliques analysis produces detailed statistics including:
\begin{itemize}
    \item Coverage metrics for DMRs, genes, and edges
    \item Size distribution of bicliques
    \item Detailed information about interesting bicliques, including:
    \begin{itemize}
        \item DMR IDs and their associated area statistics
        \item Gene names and descriptions
        \item Number of DMRs and genes in each biclique
    \end{itemize}
\end{itemize}

\subsection{Interpretation}
Interesting bicliques (≥3 DMRs, ≥3 genes) may represent important regulatory modules where multiple DMRs collectively influence multiple genes. These patterns could indicate:
\begin{itemize}
    \item Coordinated regulation of gene groups
    \item Functional relationships between genes
    \item Potential regulatory hotspots in the genome
\end{itemize}

\section{RB-Domination Analysis}
The red-blue domination analysis identifies a minimal set of DMRs (red nodes) that collectively dominate all genes (blue nodes) in the network. A gene is considered dominated if it is connected to at least one DMR in the dominating set.

\subsection{Results}
The analysis of the DSS1 dataset revealed:
\begin{itemize}
    \item \textbf{Dominating Set Size:} 1,382 DMRs (65.53\% of total DMRs)
    \item \textbf{Coverage:}
    \begin{itemize}
        \item 4,536 out of 4,536 genes dominated (100\% coverage)
        \item All genes are dominated
    \end{itemize}
    \item \textbf{Efficiency:} Each DMR in the dominating set connects to multiple genes, with examples showing individual DMRs dominating between 4-13 genes
\end{itemize}

\subsection{Interpretation}
The high percentage of genes dominated by a subset of DMRs (64.77\%) suggests:
\begin{itemize}
    \item A relatively efficient regulatory network where most genes can be reached through a subset of key DMRs
    \item Potential regulatory hotspots where individual DMRs influence multiple genes
    \item The possibility of prioritizing DMRs for experimental validation based on their domination coverage
\end{itemize}

The small number of undominated genes (20) may represent:
\begin{itemize}
    \item Genes with unique or specialized regulatory mechanisms
    \item Potential gaps in the regulatory network that warrant further investigation
\end{itemize}

This analysis complements the bicliques analysis by identifying key DMRs that could serve as regulatory hubs in the network.

Pairwise DMR graphs
The gene names naturally should relate to the same genes across all sheets, but the later timepoints will likelly have the m
ost genes. Neverthelese we should have the same node_id for the same gene name across all graphs. The DMR_id/DMR_names do NOT corresond across sheets. Ind
eed if we could use topological containment of the bicliques or minor orders between graphs to propose possible mappings of DMRs across sheets this will b
e very useful. In short we need different DMR nodes for each graph. And we could perhaps lable them using a graph subscript like DMR_{P21}7 rather than ju
st DMR7 it may help avoid confusion.
\end{document}
